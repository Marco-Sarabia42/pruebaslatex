\documentclass[a4paper,12pt]{article}
\usepackage{geometry}
\usepackage{graphicx}
\usepackage{amssymb}
\usepackage{amsmath}
\usepackage{amsthm}
\usepackage{empheq}
\usepackage{mdframed}
\usepackage{booktabs}
\usepackage{lipsum}
\usepackage{graphicx}
\usepackage{color}
\usepackage{psfrag}
\usepackage{pgfplots}
\usepackage{bm}
\usepackage{fancyhdr}
\usepackage[utf8]{inputenc}
\usepackage[spanish]{babel}
\usepackage{datetime}
\usepackage{xcolor}
\usepackage{eso-pic}

\definecolor{ocre}{RGB}{243,102,25}
\definecolor{mygray}{RGB}{243,243,244}
\definecolor{deepGreen}{RGB}{26,111,0}
\definecolor{shallowGreen}{RGB}{235,255,255}
\definecolor{deepBlue}{RGB}{61,124,222}
\definecolor{shallowBlue}{RGB}{235,249,255}

\newcommand\orangebox[1]{\fcolorbox{ocre}{mygray}{\hspace{1em}#1\hspace{1em}}}

\newtheoremstyle{mytheoremstyle}{3pt}{3pt}{\normalfont}{0cm}{\rmfamily\bfseries}{}{1em}{{\color{black}\thmname{#1}~\thmnumber{#2}}\thmnote{\,--\,#3}}
\newtheoremstyle{myproblemstyle}{3pt}{3pt}{\normalfont}{0cm}{\rmfamily\bfseries}{}{1em}{{\color{black}\thmname{#1}~\thmnumber{#2}}\thmnote{\,--\,#3}}
\theoremstyle{mytheoremstyle}
\newmdtheoremenv[linewidth=1pt,backgroundcolor=shallowGreen,linecolor=deepGreen,leftmargin=0pt,innerleftmargin=20pt,innerrightmargin=20pt,]{theorem}{Theorem}[section]
\theoremstyle{mytheoremstyle}
\newmdtheoremenv[linewidth=1pt,backgroundcolor=shallowBlue,linecolor=deepBlue,leftmargin=0pt,innerleftmargin=20pt,innerrightmargin=20pt,]{definition}{Definition}[section]
\theoremstyle{myproblemstyle}
\newmdtheoremenv[linecolor=black,left
margin=0pt,innerleftmargin=10pt,innerrightmargin=10pt,]{problem}{Problem}[section]

\usepgfplotslibrary{colorbrewer}
\pgfplotsset{width=8cm,compat=1.9}

\AddToShipoutPictureBG{%
    \includegraphics[width=\paperwidth,height=\paperheight]{/home/sarabia/pruebaslatex/imagenes/encabezado.png} % Ruta de la imagen de fondo
}

\begin{document}

\begin{flushleft}
\end{flushleft}

\begin{flushright}
    \textcolor{blue}{\textbf{COTIZACIÓN DE ESTUDIOS DE LABORATORIO}} \\
    Ciudad de México, CDMX, \today 
\end{flushright}
    

    \begin{flushleft} 
        \textbf{\sffamily  Estimado Cliente:}
        \newline  % Nueva línea
        \sffamily  % Cambiar a fuente sin serifas
        Le agradecemos su preferencia y nos complace presentar la siguiente cotización para los estudios de laboratorio que ha solicitado. 
    \end{flushleft}


    %Este apartado es para la creacion de la tabla
    %la cual esta centrada 
    \begin{center}
        \begin{tabular}{|c|c|} 
        \hline 
        \textbf{PRUEBA} &  \textbf{COSTO UNITARIO} 
        \\ \hline 
         este es un ejemplo de como se veria las celdas al rellarlas &    \\ \hline
             &    \\ \hline
             &    \\ \hline
             &    \\ \hline
             &    \\ \hline
             &    \\ \hline
        \end{tabular}
        \end{center}
    
        \begin{flushleft}
        \end{flushleft}



    \begin{flushleft}
        \sffamily  % Cambiar a fuente sin serifas
        El costo total por los estudios es de $\_\_\_\_\_$ . Sin embargo y por promocion ya que solicita diferentes estudios le estariammos cobrando $\_\_\_\_\_$ .  Este costo final, incluye
        la realizacion de los estudios de laboratorio, toma de muestra a domicilio(sin costo adicional, independientes del lugar que nos indique) e informe de reultados (detallado y en formato digitall(correo electronico y WhatsApp)).\newline

        Por otro lado, el tiempo de entrega de los resultados se estaria realizando el mismo dia de la toma de muestra. \newline
        La presente cotización tiene una vigencia de \_\_\_ días naturales a partir de la fecha de emisión.
    \end{flushleft}


    \begin{flushleft}
    \end{flushleft}

    \begin{center}
        \textcolor{blue}{\textbf{\sffamily Consulta a tu médico, es el único facultado para indicar el tipo de estudio a realizar en cada paciente.}}
    \end{center}


\end{document}